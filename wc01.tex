\documentclass[a4paper]{exam}

\usepackage{geometry}
\usepackage{graphicx}
\usepackage{hyperref}
\usepackage{titling}

\printanswers

\title{Weekly Challenge 01: Comparison}
\author{CS/MATH 113 Discrete Mathematics}
\date{Spring 2024}

\qformat{{\large\bf \thequestion. \thequestiontitle}\hfill}
\boxedpoints

\begin{document}
\maketitle

\begin{questions}
  
\titledquestion{How about them apples?}
  \begin{minipage}{.3\linewidth}
  \centerline{\includegraphics[width=\textwidth]{picard}}
\end{minipage}
\begin{minipage}{.65\linewidth}
  The \href{https://en.wikipedia.org/wiki/Replicator_(Star_Trek)}{replicator} aboard USS Enterprise has developed a fault---synthesized apples have insufficient nutrition but are otherwise identical to regular apples. Doctor \href{https://memory-alpha.fandom.com/wiki/Beverly_Crusher}{Beverly Crusher} is on the case. Scanning a bunch of apples, her \href{https://en.wikipedia.org/wiki/Medical_tricorder}{tricorder} can indicate if the bunch contains any faulty apples, but it cannot identify them.
\end{minipage}
\begin{parts}
  \part Dr. Crusher is investigating a bunch of 5 apples out of which 1 is known to be faulty. Describe how she can identify the faulty apple in no more than 3 tricorder scans.
  \part What is the minimum number of scans that Dr. Crusher needs to perform in order to guarantee finding the single faulty apple in a bunch of size $n$? Justify your answer.
\end{parts}

\begin{solution}
    % Enter your solution here.
\end{solution}

\end{questions}

\end{document}

%%% Local Variables:
%%% mode: latex
%%% TeX-master: t
%%% End:
